
\begin{table}[htbp]
\centering
\caption{\textbf{Characteristics of universal failure cases.} Cases in which all five models failed to include the reference diagnosis within the top-5 differential (\emph{n} = 6). These cases represent diagnostically challenging scenarios that exceeded current LLM capabilities regardless of model architecture.}
\label{tab:universal_failures}
\vspace{0.5em}
\small
\begin{tabular}{clllp{3cm}}
\toprule
\textbf{Case} & \textbf{Reference Diagnosis} & \textbf{Domain} & \textbf{Source} & \textbf{Potential Difficulty Factors} \\
\midrule
    18 & Central sleep apnea secondary to mandibu... & Neurology & Springer & [To be added] \\
    24 & Severe Obstructive Sleep Apnea (OSA) in ... & Neurology & Springer & [To be added] \\
    48 & Central sleep apnea due to persistent el... & Pulmonology & Springer & [To be added] \\
    89 & Postpartum sleep deprivation & Psychiatry & Springer - A Clinica... & [To be added] \\
    91 & Hypersomnia due to sleep apnea.
 & Neurology, Psychiatry & Springer - A Clinica... & [To be added] \\
    118 & risperidone & Neurology & Elsevier - Sleep Sci... & [To be added] \\
\bottomrule
\end{tabular}
\vspace{0.5em}

\footnotesize
\textit{Note:} Potential difficulty factors may include rare diagnoses, atypical presentations, overlapping clinical features, or insufficient discriminating information in the vignette.
\end{table}
